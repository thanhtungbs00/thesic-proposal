% English abstract
\cleardoublepage
\chapter*{Tóm tắt}
\addcontentsline{toc}{chapter}{Tóm tắt}
\vspace{1.0cm}
Internet of Things (IoT) phát triển với tốc độ phi thường trong những năm qua. Dữ liệu sinh ra từ các cảm biến là vô cùng lớn và nhạy cảm, đòi hỏi phải có những cơ chế thích hợp cho việc lưu trữ và chia sẻ. Việc giải quyết vấn đề trên gặp nhiều khó khăn vì các thiết bị IoT hầu hết có khả năng xử lý kém và dễ bị tấn công, việc chia sẻ dữ liệu cũng gặp nhiều khó khăn do mô hình tập trung. Trong khi đó, Blockchain, đặc biệt với sự hỗ trợ của smart contract, nổi nên như một công nghệ của sự bảo mật, tính truy vết và mô hình phi tập trung. Các tính chất này, một khi được áp dụng, sẽ cải thiện được các điểm yếu của IoT.\par
Trong báo cáo này, nhóm đã nghiên cứu và đưa ra khái niệm, tính chất và các vấn đề liên quan của từng công nghệ. Tiếp theo, nhóm đề xuất một mô hình sử dụng Blockchain để hỗ trợ quá trình chia sẻ, lưu trữ dữ liệu IoT giữa các bên. Hệ thống đảm bảo được tính toàn vẹn, bảo mật và công bằng trong quá trình chia sẻ dữ liệu. Một hệ thống thử nghiệm cũng được hiện thực để chứng minh tính khả thi của mô hình này.
\vskip0.5cm
