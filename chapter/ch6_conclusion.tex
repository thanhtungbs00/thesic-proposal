\chapter{Tổng kết}\label{chap:conclusion}
\section{Những đóng góp chính của luận văn}
Đề tài nhóm nghiên cứu tập trung giải quyết vấn đề rất quan trọng và cấp bách trong IoT hiện nay chính là chia sẻ dữ liệu. Nhờ đó, nhóm tác giả mong muốn giải quyết được vấn đề thiếu minh bạch, toàn vẹn và bảo mật dữ liệu trong quá trình chia sẻ dữ liệu giữa các bên. Khi những thiếu sót trên được tăng cường, lợi ích cho tất cả các bên tham gia cũng được tăng lên, giảm thiểu việc đánh cắp, giả mạo dữ liệu cũng như tranh chấp trong trao đổi, chia sẻ dữ liệu. \par
Nhằm khẳng định sự hiệu quả, giải pháp đề xuất của nhóm sẽ được so sánh, đánh giá với các giải pháp hiện có. Kết quả này sẽ đóng góp công bố cho cộng đồng khoa học quốc gia và quốc tế, và nó cũng có thể là một sở hữu trí tuệ trong tương lai xa hơn.\par
Thông qua đề tài, nhóm nghiên cứu mong muốn phổ biến công nghệ
Blockchain cho cộng đồng, xây dựng được một cộng đồng Blockchain vững mạnh và giàu kiến thức, từ đó đưa Việt Nam trở thành một quốc gia nổi tiếng về lĩnh vực công nghệ Blockchain.
\section{Kết quả đạt được}
Trong suốt quá trình thực hiện Đề cương Luận Văn và Luận văn Tốt nghiệp, dưới sự hướng dẫn và giúp đỡ tận tình của tiến sĩ Phạm Hoàng Anh, nhóm nghiên cứu đã bước đầu đạt được những kết quả giúp khích lệ tinh thần của các thành viên cũng như tăng niềm tin vào tính khả thi của giải pháp đề xuất của nhóm.
\section{Hướng phát triển}
Trong phạm vi báo cáo này, mục tiêu của nhóm nghiên cứu là đề xuất và chứng minh tính khả thi của đề tài, chưa tập trung phát triển thành một ứng dụng thực tế. Tuy nhiên, trong tương lai, nhóm nghiên cứu đề xuất một số cải tiến để ứng dụng thử nghiệm hoàn thiện hơn và có thể trở thành ứng dụng trong thực tế:
    
\begin{itemize}
    \item Áp dụng mô hình vào các ứng dụng IoT thực tế như thành phố thông minh, các thiết bị chăm sóc sức khỏe thông minh, nông nghiệp thông minh,... Trong đó, nhóm  đã có dịp tiếp cận và tìm hiểu một số thông tin về nông nghiệp thông minh, mà cụ thể là ứng dụng thu thập thông tin nông sản trong quá trình phát triển để hỗ trợ truy xuất nguồn gốc.\\
    Truy xuất nguồn gốc nông sản là vấn đề luôn được quan tâm ở Việt Nam. Gần đây, với sự phát triển của Blockchain, nhiều ứng dụng hỗ trợ truy xuất nguồn gốc ra đời với sự hỗ trợ của công nghệ này. Như vậy, việc thiết kế hệ thống IoT sử dụng Blockchain hỗ trợ giai đoạn đầu của quá trình truy xuất nguồn gốc là phù hợp với xu hướng hiện tại và thực tiễn nền kinh tế nông nghiệp nước ta.
    \item Tăng khả năng bảo mật bằng cách tăng giảm thông số node cảm biến tối đa trong mạng thông qua node gateway. Ví dụ hiện đang có 2 node cảm biến trong mạng, thông số node tối đa là 2, node cảm biến tiếp theo muốn tham gia vào mạng phải chờ node gateway tăng thông số node tối đa lên mới tham gia được. Bước này giúp tránh việc có node cảm biến tình cờ có được key vào mạng liền vào mạng với ý định xấu.
    \item Hiện thực giao diện cho node gateway để dễ dàng tiến hành các thao tác cấu hình như: thiết lập chu kỳ đọc dữ liệu, chu kỳ gửi dữ liệu, địa chỉ remote server, số node cảm biến tối đa, xóa node cảm biến ra khỏi mạng,...
    \item Hiện thực giao diện người dùng lấy dữ liệu từ remote server.
    \item Hiện thức một giao thức giao tiếp giữa các node để tránh tình trạng nghẽn mạng, quy trình xử lý khi node gateway mất kết nối hoặc node cảm biến mất kết nối.
    \item Phát triển hệ thống proxy re-encryption nhằm tăng cường hiệu suất, giảm thiếu thời gian và chi phí cho việc giải mã và mã hóa dữ liệu riêng cho từng thành viên yêu cầu dữ liệu.
\end{itemize}
   