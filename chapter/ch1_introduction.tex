\chapter{Giới thiệu đề tài}\label{chapter:introduction}
\pagestyle{fancy}
\section{Lý do và động lực thực hiện đề tài}
Số lượng thiết bị IoT được Cisco dự đoán sẽ đạt 50 tỷ vào năm 2020 \cite{fiftydevices}. Chúng có thể là đèn thông minh, máy điều hòa thông minh trong nhà bạn, đến những thiết bị theo dõi sức khỏe như đồng hồ thông minh, máy đo đường huyết hay thậm chí là đèn giao thông thông minh trong thành phố. Hàng loạt công nghệ truyền tải ra đời phục vụ cho mạng máy tính nói chung và IoT nói riêng như Bluetooth, WiFi, LTE, Zigbee, Z-Wave, 6LoWPAN, NFC, GSM, LoRa, NB-IoT,...\par
Dữ liệu IoT được sinh ra, chia sẻ và phân tích để cung cấp những dịch vụ có ích cho con người. Việc quản lý không tốt có thể gây mất các thông tin nhạy cảm như tình trạng sức khỏe, thói quen sinh hoạt, mất quyền điều khiển thiết bị. Ví dụ như việc bạn mất quyền điều khiển xe khi đang ngồi trong chính xe của mình, hay đơn giản là không được vào nhà vì mất quyền kiểm soát hệ thống cửa thông minh. Vì vậy, chúng ta cần có các cơ chế quản lý truy cập và chia sẻ dữ liệu IoT một cách thích hợp. Các chương trình quản lý truy cập IoT truyền thống chủ yếu được xây dựng dựa trên các mô hình quản lý truy cập phổ biến bao gồm mô hình quản lý truy cập dựa trên vai trò (Role-based access control model - RBAC), mô hình quản lý truy cập dựa trên thuộc tính (the attribute-based access control model - ABAC), mô hình quản lý truy cập dựa trên khả năng (the capability-based access control model - CapBAC).\par
Trong RBAC, quản lý truy cập được dựa trên vai trò (ví dụ: người quản trị, khách) của các đối tượng trong một tổ chức. Bằng cách liên kết các vai trò với các quyền truy cập (ví dụ: đọc, ghi) và gán vai trò cho các đối tượng, các chương trình quản lý truy cập RBAC có thể thiết lập một mối quan hệ giữa quyền truy cập và đối tượng. ABAC thực hiện quản lý truy cập dựa trên các chính sách, kết hợp nhiều loại của các thuộc tính để xây dựng một bộ quy tắc thể hiện trong những điều kiện nào thì quyền truy cập có thể được cấp cho các đối tượng nào. Đối với CapBAC, quyền truy cập được cấp cho các đối tượng dựa trên khả năng của đối tượng đó.\par
Điều đáng chú ý ở đây là trong các cách trên, việc xác nhận quyền truy cập của các đối tượng được thực hiện bởi một thực thể tập trung. Điều này dẫn đến việc thất bại đơn điểm (single point of failure). Để giải quyết vấn đề này, các mô hình CapBAC phân tán đã được đề xuất, trong đó việc xác nhận quyền truy cập được thực hiện bởi chính các thiết bị IoT thay vì là một thực thể tập trung. Tuy nhiên, các thiết bị IoT thường có kích thước nhỏ, khả năng lưu trữ, xử lý hạn chế  và do đó có thể dễ dàng bị xâm phạm. Vì vậy, các thiết bị IoT không thể được tin tưởng hoàn toàn như một thực thể xác nhận quyền truy cập. Kết quả là các mô hình CapBAC phân tán không giải quyết được vấn đề quản lý truy cập không đáng tin cậy trong môi trường IoT. \par
% Hầu hết các thiết bị IoT có kích thước nhỏ, khả năng lưu trữ, xử lý hạn chế nên đa phần các cơ chế này được đảm nhận bởi các máy chủ tập trung. Tuy nhiên, mô hình tập trung lại có những nhược điểm lớn như tính sẵn sàng thấp, thiếu minh bạch, thiếu bảo mật, dễ bị thao túng từ bên trong và tấn công từ bên ngoài.\par
Trong khi đó, sau khi xuất hiện từ năm 2008, với sự thành công trong lĩnh vực tài chính, Blockchain đã chứng minh được các tính chất đặc biệt của mình, mà tính chất quan trọng nhất đó là sử dụng mô hình phi tập trung. Do đó, Blockchain, một khi được kết hợp với IoT, sẽ là giải pháp tốt để thay thế kiến trúc tập trung trong vấn đề quản lý dữ liệu IoT.\par
Các chuyên gia trong lĩnh vực nghiên cứu lẫn công nghiệp đều đang không ngừng nỗ lực chạy đua để tìm ra phương án kết hợp hiệu quả IoT và Blockchain. Khả năng kết hợp trên trở nên rõ ràng hơn từ khi xuất hiện smart contract (hợp đồng thông minh).\par
Trong báo cáo này, nhóm nghiên cứu đề xuất một mô hình sử dụng smart contract nhằm quản lý quá trình truy cập, chia sẻ dữ liệu IoT. 




\section{Mục tiêu và giới hạn đề tài}
Quản lý truy cập trong IoT có thể được thực hiện ở nhiều khía cạnh khác nhau. Ví dụ, quản lý quyền sở hữu thiết bị, quản lý việc công bố dữ liệu, quản lý chia sẻ dữ liệu,... Tuy nhiên, trong phạm vi báo cáo này, nhóm nghiên cứu đề xuất một hệ thống quản lý chia sẻ dữ liệu giữa những người có thiết bị sinh ra dữ liệu và những người có nhu cầu sử dụng dữ liệu. \par
Mục tiêu đề tài là xây dựng một hệ thống quản lý chia sẻ dữ liệu trong IoT thông qua smart contract nhằm tăng cường tính toàn vẹn, bảo mật của dữ liệu. Đồng thời, tất cả các giao dịch của việc chia sẻ dữ liệu giữa các thành viên trong hệ thống đều được lưu trữ minh bạch, công khai và không thể thay đổi. Từ đó, việc truy vết các giao dịch và giải quyết tranh chấp giữa các bên được diễn ra dễ dàng và chống chối bỏ. Trong giới hạn của báo cáo này, nhóm nghiên cứu tập trung hiện thực hệ thống nhằm chứng minh tính khả thi của giải pháp quản lý truy cập trong IoT sử dụng smart contract. Do đó, những vấn đề như mở rộng, chi phí giao dịch chưa được nhóm nghiên cứu xem xét đến như là một tiêu chí để đánh giá hệ thống.



\section{Cấu trúc báo cáo luận văn}
Phần còn lại của báo cáo này được tổ chức như sau. Phần 2 trình bày các kiến thức nền tảng về công nghệ Internet vạn vật (IoT), công nghệ chuỗi khối (Blockchain) và những khó khăn, thách thức khi có sự kết hợp giữa hai công nghệ trên. Phần 3 trình bày giải pháp đề xuất của nhóm nghiên cứu và kiến trúc hệ thống được thiết kế cho giải pháp này. Phần 4 đề cập đến việc hiện thực và đánh giá ưu nhược điểm của giải pháp. Trong phần 5, nhóm nghiên cứu tổng kết những đóng góp chính của luận văn, kết quả đạt được cũng như hướng phát triển của đề tài. 

